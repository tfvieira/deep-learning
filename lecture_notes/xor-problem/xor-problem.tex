\documentclass[aspectratio=169,mathserif]{beamer}
%\documentclass[aspectratio=169,mathserif,handout]{beamer}


\usepackage{graphicx} % Allows including images
\graphicspath{{./img/}}

\usepackage{epstopdf}
\epstopdfsetup{outdir=./img/}

\usepackage[utf8]{inputenc}
%Load useful packages
\usepackage{booktabs} % Allows the use of \toprule, \midrule and \bottomrule in tables
\usepackage{subcaption}
\usepackage{subfiles}
\usepackage{url}
\usepackage{amssymb}
%
\usepackage{multirow}
\usepackage{mathrsfs} 
%
% align enum description
% \usepackage{enumitem}
% refs
\usepackage[style=authoryear,natbib=true]{biblatex}
\bibliography{references.bib}
%
\usepackage{tikz}
\usepackage{tikz-cd}
\usepackage{tikzscale}
\usetikzlibrary{calc,matrix,chains,positioning,decorations.pathreplacing,decorations.text,arrows,cd}

% overlays
\tikzset{
invisible/.style={opacity=0},
visible on/.style={alt={#1{}{invisible}}},
alt/.code args={<#1>#2#3}{%
\alt<#1>{\pgfkeysalso{#2}}{\pgfkeysalso{#3}} % \pgfkeysalso doesn't change the path
},
}

% neural nets
\tikzset{%
  every neuron/.style={
circle,
draw,
%minimum size=1cm
  },
  neuron missing/.style={
draw=none, 
scale=1,
text height=0.333cm,
execute at begin node=\color{black}$\vdots$
  },
}

% infrastructure
\tikzset{
vertex/.style = {
circle,
fill= black,
outer sep = 2pt,
inner sep = 1pt,
}
}




%Information to be included in the title page:
\title{Deep Learning}

\author{Tiago Vieira}
\institute{Institute of Computing\\Universidade Federal de Alagoas}
\date{\today}
%Logo in every slide
%\logo{%
%  \makebox[0.98\paperwidth]{
%\includegraphics[width=0.5cm,keepaspectratio]{../logos/ufal_logo.png}%
%\hfill%\includegraphics[height=1cm,keepaspectratio]{logos/edge_logo.png}%
%\includegraphics[height=0.8cm,keepaspectratio]{../logos/ic_logo.png}%
%  }
%}
%Contents before every section's starting slide
\AtBeginSubsection[]
{
  \begin{frame}
\frametitle{Summary}
\scriptsize
\tableofcontents[currentsection,currentsubsection]
  \end{frame}


}
% shape, colour of item, nested item bullets in itemize only
% \setbeamertemplate{itemize item}[square] \setbeamercolor{itemize item}{bg=blue}
% \setbeamertemplate{itemize subitem}[circle] \setbeamercolor{itemize subitem}{fg=green}
% \setbeamertemplate{itemize subsubitem}[triangle] \setbeamercolor{itemize subsubitem}{fg=red}
% font size of nested and nested-within-nested bulltes in both itemize and enumerate
% options are \tiny, \small, \scriptsize, \normalsize, \footnotesize, \large, \Large, \LARGE, \huge and \Huge
\setbeamerfont{itemize/enumerate subbody}{size=\scriptsize} 
\setbeamerfont{itemize/enumerate subsubbody}{size=\scriptsize}
% figure numbers
% \setbeamertemplate{caption}[numbered]
% blocks style
\setbeamertemplate{blocks}[rounded][shadow=true]


% Hide nav control
\usenavigationsymbolstemplate{}
% add numbering
%\addtobeamertemplate{navigation symbols}{}{%
%\usebeamerfont{footline}%
%\usebeamercolor[fg]{footline}%
%\hspace{1em}%
%\insertframenumber/\inserttotalframenumber
%}

% Footnote without number
\newcommand\blfootnote[1]{%
\begingroup
\renewcommand\thefootnote{}\footnote{#1}%
\addtocounter{footnote}{-1}%
\endgroup
}

% items symbols
\setbeamertemplate{itemize subitem}{-}
\setbeamertemplate{itemize subsubitem}{-}

%%setting up some useful slide creation commands
%split slide
% \newenvironment{splitframe}[5]
% %[1] ==> 1 parameter passed through {}
% %[2] ==> 2 parameters passed through {}{}
% %[4] ==> 4 parameters passed through {}{}{}{}
% {
% \begin{frame}{#3}
% \begin{columns}
% \column{#1\linewidth}
% \centering
% #4
% \column{#2\linewidth}
% \centering
% #5
% \end{columns}
% \centering
% \vspace{\baselineskip} % adds one line space
% }
% %Inside the first pair of braces (ABOVE) is set what your new environment will do before the text within, then inside the second pair of braces (BELOW) declare what your new environment will do after the text. Note second pair can be empty braces too.
% {
% \end{frame}


% }

%\usepackage{calligra}
%\DeclareMathAlphabet{\mathcalligra}{T1}{calligra}{m}{n}


\usepackage{svg}
\usepackage{PSTricks}

\subtitle{The XOR Problem}

\begin{document}

\frame{\titlepage}

\section{The XOR Problem}


\begin{frame}
\frametitle{The XOR Problem}
\begin{itemize}
\item ``\textit{Perceptrons}'' by Marvin Minsky and Seymour Papert (1969).
\item Perceptrons cannot solve the XOR problem.
\item Significant decline in interest and funding of neural network research.
\end{itemize}
\end{frame}


\begin{frame}
\frametitle{The XOR Problem}
\begin{center}
\includegraphics[scale=.4]{xor-a.png}
\end{center}
\end{frame}



\begin{frame}
\frametitle{Rectified Linear Activation}
\begin{center}
\includegraphics[width=.6\textwidth]{relu.png}
\end{center}
\end{frame}




\begin{frame}
\frametitle{Network Diagrams}
\begin{columns}
\begin{column}{.4\textwidth}
\centering
\begin{center}
\includegraphics[scale=.2]{network-diagrams.png}
\end{center}
$
\textbf{h} = \max(0, \textbf{W}^{T}\textbf{x} + \textbf{c})
$\\
$
f(\textbf{x}; (\textbf{W}; \textbf{c}); (\textbf{w}, b)) = \textbf{w}^{T} \textbf{h} + b
$
\end{column}
\begin{column}{.6\textwidth}
\end{column}
\end{columns}
\end{frame}


\begin{frame}
\frametitle{Solving XOR}
\begin{columns}
\begin{column}{.4\textwidth}
\centering
\begin{center}
\includegraphics[scale=.2]{network-diagrams.png}
\end{center}
$
\textbf{h} = \max(0, \textbf{W}^{T}\textbf{x} + \textbf{c})
$\\
$
f(\textbf{x}; (\textbf{W}; \textbf{c}); (\textbf{w}, b)) = \textbf{w}^{T} \textbf{h} + b
$
\end{column}
\begin{column}{.6\textwidth}
$
X = \left [
\textbf{x}
\right ]_{i=1}^{4}
=
\left [
\begin{array}{c}
x_1 \\
x_2
\end{array}
\right ]
=
\left [
\begin{array}{cccc}
0 & 1 & 0 & 1 \\
0 & 0 & 1 & 1
\end{array}
\right ]
$\\
$
W = 
\left [
\begin{array}{cc}
1 & 1 \\
1 & 1
\end{array}
\right ]
$\\
$
\textbf{c} =
\left [
\begin{array}{c}
0 \\
-1
\end{array}
\right ]
$\\
$
\textbf{w} =
\left [
\begin{array}{c}
1 \\
-2
\end{array}
\right ]
$\\
$
b = 0
$
\end{column}
\end{columns}
\end{frame}


\begin{frame}
\frametitle{Solving XOR}
\begin{columns}
\begin{column}{.4\textwidth}
\centering
\begin{center}
\includegraphics[scale=.2]{network-diagrams.png}
\end{center}
$
\textbf{h} = \max(0, \textbf{W}^{T}\textbf{x} + \textbf{c})
$\\
$
f(\textbf{x}; (\textbf{W}; \textbf{c}); (\textbf{w}, b)) = \textbf{w}^{T} \textbf{h} + b
$
\end{column}
\begin{column}{.6\textwidth}
$
H = 
\max \left (
0,
\textbf{W}^T X + \textbf{c}
\right )
$\\

$
H = 
\max \left (0, 
\left [
\begin{array}{cc}
1 & 1 \\
1 & 1
\end{array}
\right ]
\left [
\begin{array}{cccc}
0 & 1 & 0 & 1 \\
0 & 0 & 1 & 1
\end{array}
\right ]
+
\left [
\begin{array}{c}
0 \\
-1
\end{array}
\right ]
\right )
$\\
$
H = 
\max \left (0, 
\left [
\begin{array}{cccc}
0 & 1 & 1 & 2 \\
0 & 1 & 1 & 2
\end{array}
\right ]
+
\left [
\begin{array}{c}
0 \\
-1
\end{array}
\right ]
\right )
$\\
%$
%H = 
%\max \left (0, 
%\left [
%\begin{array}{cccc}
%0 & 1 & 1 & 2 \\
%-1 & 0 & 0 & 1
%\end{array}
%\right ]
%\right )
%$\\
$
H = 
\max \left (0, 
\left [
\begin{array}{cccc}
0 & 1 & 1 & 2 \\
-1 & 0 & 0 & 1
\end{array}
\right ]
\right )
$\\
$
H =
\left [
\begin{array}{cccc}
0 & 1 & 1 & 2 \\
0 & 0 & 0 & 1
\end{array}
\right ]
$
\end{column}
\end{columns}
\end{frame}

%\begin{frame}
%
%\end{frame}

\begin{frame}
\begin{center}
\includegraphics[scale=.35]{h.png}
\end{center}
\end{frame}


\begin{frame}
\begin{columns}[t]
\begin{column}{.4\textwidth}
\begin{center}
\includegraphics[scale=.2]{network-diagrams.png}
\end{center}
$
\textbf{h} = \max(0, \textbf{W}^{T}\textbf{x} + \textbf{c})
$\\
$
f(\textbf{x}; (\textbf{W}; \textbf{c}); (\textbf{w}, b)) = \textbf{w}^{T} \textbf{h} + b
$
\end{column}
\begin{column}{.6\textwidth}
\[
Y = \max
\left (
0,
\textbf{w}^T H + \textbf{b}
\right )
\]
\[
Y = \max
\left (
\left [
\begin{array}{cc}
1 & -2
\end{array}
\right ]
\left [
\begin{array}{cccc}
0 & 1 & 1 & 2 \\
0 & 0 & 0 & 1
\end{array}
\right ]
+
\left [
\begin{array}{c}
0 \\
0
\end{array}
\right ]
\right )
\]
\[
Y =
\left [
\begin{array}{cccc}
0 & 1 & 1 & 0
\end{array}
\right ]
\]
\end{column}
\end{columns}
\end{frame}

%\begin{frame}
%Rectified Linear Activation (ReLU)
%\begin{itemize}
%\item Applying this function to the output of a linear transformation yields a nonlinear transformation.
%\item Very close to linear.
%\item Very simple nonlinearity (2 pieces -- piecewise-linear).
%\item Sufficient to represent any function if enough hidden units are connected.
%\item Default activation function recommended for use with most feedforward NNs.
%\end{itemize}
%Why ReLU is so effective?
%\begin{itemize}
%\item Strong gradient. Gradient descent can compute large gradients.
%\item Consistent behavior across its whole domain.
%\item Historical reasons.
%\end{itemize}
%\end{frame}



\begin{frame}
\frametitle[alignment=center]{}
\flushbottom
\centering
Thank you!\\
\href{mailto:tiago@ic.ufal.br}{tvieira@ic.ufal.br}\\
%\href{mailto:warley.barbosa@edge.ufal.br}{warley.barbosa@edge.ufal.br}\\
%\href{mailto:icaro.bastos@edge.ufal.br}{icaro.bastos@edge.ufal.br}\\
\end{frame}



\end{document}
